\documentclass[12pt]{article}
\usepackage{hyperref}
\usepackage{termcal}
\usepackage{enumitem}
\usepackage{color}
\textwidth=7in
\textheight=9.5in
\topmargin=-1in
\headheight=0in
\headsep=.5in
\hoffset  -.85in

\pagestyle{empty}

\renewcommand{\thefootnote}{\fnsymbol{footnote}}
\begin{document}

\begin{center}
{\bf AECN 896-002\   
}
\end{center}

\setlength{\unitlength}{1in}

\begin{picture}(6,.1) 
\put(0,0) {\line(1,0){6.25}}         
\end{picture}

 \newcommand{\MWFClass}{%
\calday[Monday]{\classday} % Monday
\skipday % Tuesday (no class)
\calday[Wednesday]{\classday} % Wednesday
\skipday % Thursday (no class)
\calday[Friday]{\classday} % Friday 
\skipday\skipday % weekend (no class)
}

\newcommand{\TRClass}{%
\skipday % Monday (no class)
\calday[Tuesday]{\classday} % Tuesday
\skipday % Wednesday (no class)
\calday[Thursday]{\classday} % Thursday
\skipday % Friday 
\skipday\skipday % weekend (no class)
}

\newcommand{\Holiday}[2]{%
\options{#1}{\noclassday}
\caltext{#1}{#2}
}

\SetLabelAlign{parright}{\parbox[t]{\labelwidth}{\raggedleft#1}}

\setlist[description]{style=multiline,topsep=10pt,leftmargin=5cm,font=\normalfont,align=parleft}

\renewcommand{\arraystretch}{2}

%======================================================
% Instructor
%======================================================
\vskip.25in

\noindent\textbf{Instructor:} Taro Mieno
\begin{itemize}
	\item  E-mail: tmieno2@unl.edu 
\end{itemize} 

%======================================================
% TA
%======================================================
\vskip.25in

\noindent\textbf{TA:} Shunkei Kakimoto
\begin{itemize}
	\item  E-mail: skakimoto3@huskers.unl.edu 
\end{itemize} 

\vskip.15in

%======================================================
% Lecture, Labs, Office Hours
%======================================================
\noindent\textbf{Lectures and Labs:} 
\begin{itemize}
	\item Lectures: MW 3:00 - 4:30 PM (zoom)
	\item Labs: F 1:00 - 2:30 PM (zoom)
\end{itemize}

zoom link: \url{https://unl.zoom.us/j/93493145774}
\vskip.15in

\noindent\textbf{Office Hours:} Wednesday, 10:00 to 11:30 pm or by appointment\\

\noindent\textbf{Course Description:} The main goal of this course is to learn how to conduct empirical research fairly independently by the end of the semester. In order to achieve this goal, students will be introduced to basic econometric theories through lectures. Further, students will be given plenty opportunities to apply econometric theories to actual empirical problems both during lectures and through assignments. Laboratory sessions lead primarily by TAs are designed so students learn how to use statistical software to conduct econometric analysis independently, along with data management and visualization. 

%======================================================
% Reading Materials
%======================================================
\vskip.15in
\noindent\textbf{Reading Materials:}\\

\noindent \underline{Required}: Wooldridge, Jeffrey M. 2006. ``Introductory Econometrics: A Modern Approach (5th edition).'' Mason, OH: Thomson/South-Western.\\
\vskip.02in
\noindent \underline{Recommended}: Florian, Heiss. 2016 ``Using R for Introductory Econometrics.'' CreateSpace Independent Publishing Platform. \\
\vskip.02in
\noindent \underline{Recommended}: Norman, Matloff. 2011 ``The Art of R Programming: A Tour of Statistical Software Design.'' No Starch Press. \\
\vskip.02in

\noindent\textbf{Prerequisites:} Intermediate calculus and statistics

\newpage 
\noindent\textbf{Grading:}
\begin{center}
\begin{tabular}{lc}
	 Assignments (5 assignments) :& 50\% \\
	 Paper: & 50\%\\
	   $\;\;\;\;$ Proposal: & 5\%\\
	   $\;\;\;\;$ Final paper: & 45\%\\\hline
	 Total: & 100\%
\end{tabular}
\end{center}

\begin{itemize}
	\item \textbf{Assignments:} There will be 4 assignments. Late submissions will have 1/3 of a letter grade deducted from the grade for that submission, increasing by an additional 1/3 grade for each 24 hours beyond the deadline.
	\item \textbf{Paper:} This assignment consists of two parts: proposal and final paper. 

	\begin{itemize}
		\item \textbf{Proposal}:
			\begin{description}
			\item [Objective:] Brief descriptions of the objective of the final paper 
			\item [Datasets:] Brief descriptions of datasets you will use 
			\end{description}
		\item \textbf{Final paper}: 	
		\begin{description}
			\item [Introduction:] 
			\begin{enumerate}
				\item clear identification of what you are trying to find out (research question) [1 point]
				\item why the research question is worthwhile answering [1 point]
			\end{enumerate}
			\item [Data description:] 
			\begin{enumerate}
				\item the nature of the data with summary statistics table [1 point]
				\item visualize a few key variables in a meaningful way [3 points]
			\end{enumerate}
			\item [Econometric methods:] the \textcolor{blue}{process} of how you end up with the final econometric models and methods. [40 points (\textcolor{blue}{or more})]   
			\begin{enumerate}
				\item justification of your choice of independent variables 
				\item potential endogeneity problems 
				\item what did you do to address the endogeneity problems?
				\item justification of econometric model(s) and method(s)
			\end{enumerate}
			\item [Results and discussions:] 
				\begin{enumerate}
					\item interpret and describe the results [2 points]
					\item implications of the results [1 point]
				\end{enumerate}
			\item [Conclusions] conclusions [1 point]
		\end{description}
	\end{itemize}

	You write a paper with a particular emphasis on econometric analysis using a real world data set (\textcolor{red}{due: May, 11}). You are encouraged to use the datasets you are using for your masters thesis (talk with your advisor). Otherwise, you must find and use a \textcolor{red}{panel} data set. 

	You also write a paper proposal for your final paper (\textcolor{red}{due: April, 1}). This assignment is for keeping you on track for making timely progress on your final paper. Before you write a proposal, you will need to consult with me for your research topic and datasets to be approved (\textcolor{red}{due: March, 23}). This ensures that your final paper is feasible.  

	You present your paper proposal in class. Presentations are not graded according to the content of your paper, rather on your presentation skills. Here is the timeline of the paper assignment:

  You may be able to find a panel dataset that fits your interest from the following sources (not all datasets are panel datasets):

  \begin{itemize}
    \item \href{https://www.openicpsr.org/openicpsr/search/studies}{OPENICPSR}
    \item \href{https://blog.repec.org/2020/08/04/a-replication-database-for-economics-and-social-sciences-the-replicationwiki/}{Replication Wiki}
    \item \href{https://cran.r-project.org/web/packages/wooldridge/wooldridge.pdf}{woolrdridge R package}
  \end{itemize}
  

\end{itemize}

\vspace*{.15in}

%===================================
% Schedule
%===================================
\clearpage
\paragraph*{Tentative Schedule:}
\begin{center}

% Semester starts on 1/11/2010 and last for 16
% weeks, including finals week

\begin{calendar}{1/17/2022}{16}
\setlength{\calboxdepth}{.4in}
\setlength{\calwidth}{7in}
\MWFClass

%--------------------------
% Schedule
%--------------------------
%--- week 1 ---%
\caltextnext{Introduction to econometrics}
\caltextnext{\textcolor{blue}{Lab 1}: Introduction to R}

%--- week 2 ---%
\caltextnext{Simple univariate regression}
\caltextnext{Simple univariate regression}
\caltextnext{\textcolor{blue}{Lab 2}: Rmarkdown}

%--- week 3 ---%
\caltextnext{Simple univariate regression}
\caltextnext{Monte Carlo simulation}
\caltextnext{\textcolor{blue}{Lab 3}: Assignment 1 review \textcolor{red}{Assignment 1 due before the class}}

%---  ---%

\caltextnext{Multivariate regression}
\caltextnext{Multi-collinearity and omitted variable}
\caltextnext{\textcolor{blue}{Lab 4}: Data management I (dplyr)}

\caltextnext{Inference}
\caltextnext{Heteroskedasticity and robust standard error estimation}
\caltextnext{\textcolor{blue}{Lab 5}: Data management II (dplyr)}

\caltextnext{Clustered error and bootstrap}
\caltextnext{Functional form and scaling}
\caltextnext{\textcolor{blue}{Lab 6}: Assignment 2 review\\
 \textcolor{red}{Assignment 2 due before class}}

\caltextnext{Dummy variables}
\caltextnext{Panel data methods}
\caltextnext{\textcolor{blue}{Lab 7}: data visualization 1}

\caltextnext{Panel data methods}
\caltextnext{Panel data methods}
\caltextnext{\textcolor{blue}{Lab 8}: data visualization 2}

\caltextnext{Panel data methods}
\caltextnext{Panel data methods and paper expectation}
\caltextnext{\textcolor{blue}{Lab 9}: Assignment 3 review\\
 \textcolor{red}{Assignment 3 due before class}}

\caltextnext{Causal Inference}
\caltextnext{Causal Inference}
\caltextnext{\textcolor{blue}{Lab 10}: Research flow and R I (research question identification and data collection)}

\caltextnext{Causal Inference}
\caltextnext{Causal Inference}
\caltextnext{\textcolor{blue}{Lab 11} Research flow and R II (data management)}

\caltextnext{Causal Inference}
\caltextnext{Causal Inference}
\caltextnext{\textcolor{blue}{Lab 12}: Assignment 4 review \\ \textcolor{red}{Assignment 4 due}}

\caltextnext{Causal Inference}
\caltextnext{Causal Inference}
\caltextnext{\textcolor{blue}{Lab 13} Research flow and R III (exploratory analysis)}

\caltextnext{Limited dependent variable}
\caltextnext{Limited dependent variable}
\caltextnext{\textcolor{blue}{Lab 13} Research flow and R IV (regression analysis and reporting)}

\caltextnext{Limited dependent variable}
\caltextnext{Limited dependent variable}
\caltextnext{\textcolor{red}{No Class}}

% Limited dependent variable (count)
% Survival analysis 
% \textcolor{blue}{Lab 13}: Survival 
% Semi-parametric regression

%--------------------------
% Holidays
%--------------------------
\Holiday{3/14/2022}{\textcolor{red}{Spring break: No class}}
\Holiday{3/16/2022}{\textcolor{red}{Spring break: No class}}
\Holiday{3/18/2022}{\textcolor{red}{Spring break: No class}}

% %--------------------------
% % Finals week
% %--------------------------
% \options{4/26/2010}{\noclassday} % finals week
% \options{4/27/2010}{\noclassday} % finals week
% \options{4/28/2010}{\noclassday} % finals week
% \options{4/29/2010}{\noclassday} % finals week
% \options{4/30/2010}{\noclassday} % finals week
% \caltext{4/27/2010}{\textbf{Final Exam}}
\end{calendar}
\end{center}
\vskip.25in

\newpage   

\vskip.25in
\noindent\textbf{Academic Honesty:}\\ 
\noindent Students are expected to adhere to guidelines concerning academic dishonesty outlined in Section 4.2 of University's Student Code of Conduct (\url{http://stuafs.unl.edu/ja/code/}). Students are encouraged to contact the instructor for clarification of these guidelines if they have questions or concerns. The Department of Agricultural Economics has a written policy defining academic dishonesty, the potential sanctions for incidents of academic dishonesty, and the appeal process for students facing potential sanctions. The Department also has a policy regarding potential appeals of final course grades. These policies are available for review on the department’s website (\url{http://agecon.unl.edu/undergraduate})\\

\vskip.25in
\noindent\textbf{Students with disabilities:}\\
\noindent Students with disabilities are encouraged to contact the instructor for a confidential discussion of their individual needs for academic accommodation. It is the policy of the University of Nebraska-Lincoln to provide flexible and individualized accommodation to students with documented disabilities that may affect their ability to fully participate in course activities or to meet course requirements. To receive accommodation services, students must be registered with the Services for Students with Disabilities (SSD) office, 132 Canfield Administration, 472-3787 voice or TTY. 

\vskip.25in
\noindent\textbf{Mask Requirement:}\\
\noindent At the moment, the university policy states that students are required to wear a mask in the classroom. 



\end{document}