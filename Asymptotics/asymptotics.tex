\documentclass[fleqn]{beamer}\usepackage[]{graphicx}\usepackage[]{color}
%% maxwidth is the original width if it is less than linewidth
%% otherwise use linewidth (to make sure the graphics do not exceed the margin)
\makeatletter
\def\maxwidth{ %
  \ifdim\Gin@nat@width>\linewidth
    \linewidth
  \else
    \Gin@nat@width
  \fi
}
\makeatother

\definecolor{fgcolor}{rgb}{0.345, 0.345, 0.345}
\newcommand{\hlnum}[1]{\textcolor[rgb]{0.686,0.059,0.569}{#1}}%
\newcommand{\hlstr}[1]{\textcolor[rgb]{0.192,0.494,0.8}{#1}}%
\newcommand{\hlcom}[1]{\textcolor[rgb]{0.678,0.584,0.686}{\textit{#1}}}%
\newcommand{\hlopt}[1]{\textcolor[rgb]{0,0,0}{#1}}%
\newcommand{\hlstd}[1]{\textcolor[rgb]{0.345,0.345,0.345}{#1}}%
\newcommand{\hlkwa}[1]{\textcolor[rgb]{0.161,0.373,0.58}{\textbf{#1}}}%
\newcommand{\hlkwb}[1]{\textcolor[rgb]{0.69,0.353,0.396}{#1}}%
\newcommand{\hlkwc}[1]{\textcolor[rgb]{0.333,0.667,0.333}{#1}}%
\newcommand{\hlkwd}[1]{\textcolor[rgb]{0.737,0.353,0.396}{\textbf{#1}}}%
\let\hlipl\hlkwb

\usepackage{framed}
\makeatletter
\newenvironment{kframe}{%
 \def\at@end@of@kframe{}%
 \ifinner\ifhmode%
  \def\at@end@of@kframe{\end{minipage}}%
  \begin{minipage}{\columnwidth}%
 \fi\fi%
 \def\FrameCommand##1{\hskip\@totalleftmargin \hskip-\fboxsep
 \colorbox{shadecolor}{##1}\hskip-\fboxsep
     % There is no \\@totalrightmargin, so:
     \hskip-\linewidth \hskip-\@totalleftmargin \hskip\columnwidth}%
 \MakeFramed {\advance\hsize-\width
   \@totalleftmargin\z@ \linewidth\hsize
   \@setminipage}}%
 {\par\unskip\endMakeFramed%
 \at@end@of@kframe}
\makeatother

\definecolor{shadecolor}{rgb}{.97, .97, .97}
\definecolor{messagecolor}{rgb}{0, 0, 0}
\definecolor{warningcolor}{rgb}{1, 0, 1}
\definecolor{errorcolor}{rgb}{1, 0, 0}
\newenvironment{knitrout}{}{} % an empty environment to be redefined in TeX

\usepackage{alltt}

% \usefonttheme[onlylarge]{structuresmallcapsserif}
\usefonttheme[onlysmall]{structurebold}
\usepackage[normalem]{ulem} % use normalem to protect \emph
\newcommand\hl{\bgroup\markoverwith
  {\textcolor{yellow}{\rule[-.5ex]{2pt}{2.5ex}}}\ULon}
\usepackage{amsmath,amssymb}
\usepackage{booktabs}
\usepackage{graphicx}
\usepackage{longtable}
\usepackage{lscape}
\usepackage[authoryear]{natbib}
\usepackage{mdframed}
\usepackage{array}
\usepackage{listings}
\usepackage{wasysym}
\usepackage{url}
\usepackage{multirow}
\usepackage{color}
\usepackage{qtree}
\usepackage{animate}
\usepackage{inconsolata}
\usepackage{framed}
\definecolor{shadecolor}{rgb}{1,0.8,0.3}
\usepackage[justification=centering]{caption}
\captionsetup{compatibility=false}
\usepackage{subcaption}
\usepackage{xcolor}
\usepackage{hyperref}
\hypersetup{colorlinks=true,linkcolor=blue}
\usepackage[space]{grffile}
\usefonttheme[onlymath]{serif}
% \usepackage{tikz}
% \usetikzlibrary{calc,matrix}
\usepackage{sgame}
\usepackage{tikz}
\usetikzlibrary{trees}

\mode<presentation>
%\usetheme{Frankfurt}
\usetheme{boxes}
\usecolortheme{beaver}
%\usetheme{Singapore}
%\usetheme{Hannover}

\newcommand*{\captionsource}[2]{%
  \caption[{#1}]{%
    #1%
    \\\hspace{\linewidth}%
    {\small\textbf{Source:} #2}%
  }%
}

\newenvironment{changemargin}[2]{% 
  \begin{list}{}{% 
    \setlength{\topsep}{0pt}% 
    \setlength{\leftmargin}{#1}% 
    \setlength{\rightmargin}{#2}% 
    \setlength{\listparindent}{\parindent}% 
    \setlength{\itemindent}{\parindent}% 
    \setlength{\parsep}{\parskip}% 
  }% 
  \item[]}{\end{list}} 

%Then \begin{changemargin}{-1cm}{-1cm} makes the margin 1 cm wider on either side of the page until \end{changemargin} appears. 

\beamertemplatenavigationsymbolsempty
\setbeamertemplate{blocks}[rounded]
\newenvironment<>{block_code}[1]{%
  \setbeamercolor{block title}{fg=white,bg=black}%
  \begin{block}#2{#1}}{\end{block}}

%--------------------------
% Reduce the spacing between R code and output
%--------------------------
\usepackage{etoolbox} 
\makeatletter 
\preto{\@verbatim}{\topsep=0pt \partopsep=0pt } 
\makeatother

\renewenvironment{knitrout}{\setlength{\topsep}{0mm}}{} 

%===================================
% Coloring change
%===================================
\definecolor{darkred}{rgb}{0.8,0,0}
\setbeamercolor{block title}{fg=darkred,bg=structure.fg!20!bg!50!bg}
\setbeamercolor{block body}{use=block title,bg=block title.bg}

\title{OLS Asymptotics}
\author{Taro Mieno}
\date{AECN 896-003: Applied Econometrics}
\everymath{\displaystyle}
\IfFileExists{upquote.sty}{\usepackage{upquote}}{}
\begin{document}
\begin{frame}
\titlepage
\end{frame}



\begin{frame}[c,fragile]
  \frametitle{OLS Asymptotics (Large Sample Properties)}
  
  \begin{block}{What is it?}  
  	\begin{itemize}
  		\item Properties of OLS that hold only when the sample size is infinite (\textcolor{blue}{very} large) 
  		\only<2->{\item (loosely put) How OLS estimators behave when the number of observations goes \textcolor{blue}{infinite (really large)}}
  	\end{itemize}
  \end{block}

  \only<3->{\begin{block}{Small sample properties}  
  Under certain conditions,
    \begin{itemize}
      \item Unbiasedness of OLS estimators
      \item Efficiency of OLS estimators
    \end{itemize} 
  hold \textcolor{blue}{whatever the sample size is} (including infinite numbers of observations).
  \end{block}}
\end{frame}

%===================================
% Consistency
%===================================
\title{Consistency}
\author{}
\date{}

\begin{frame}
\maketitle
\end{frame}

\begin{frame}[c]
	\frametitle{Consistency}
	\begin{block}{Verbally (and very loosely),}	
		An estimator is \textcolor{blue}{consistent} if the probability that the estimator produces the true parameter is 1 when sample size is infinite. 
	\end{block}	

	\only<2->{
	\begin{block}{Example:}	
	OLS estimator of the coefficient on $x$ in the following model with all $MLR.1$ through $MLR.4$ satisfied:
		\begin{align*}
			y_i = \beta_0 + \beta_1 x_i + u_i
		\end{align*} 
	with all the conditions necessary for the unbiasedness property of OLS satisfied.
	\end{block}
	}
\end{frame}

\begin{frame}[c,fragile]
  \frametitle{MC simulations: consistency of OLS estimators}
  \begin{block}{Conceptual steps of MC simulations}  
    \begin{itemize}
    \item generate data ($N$ observations) according to $y_i = \beta_0 + \beta_1 x_i + u_i$
    \item run on the generated data
    \item store the coefficient estimate
    \item repeat the above experiment 1000 times
    \item examine how the coefficient estimates are distributed
  \end{itemize}
  \end{block}
  \only<2->{\begin{block}{What you should see is}  
    As $N$ gets larger (more observations), the distribution of $\hat{\beta}_1$ get more tightly centered around its true value (here, $1$)
  \end{block}} 
\end{frame}

\begin{frame}[c,fragile]
  \frametitle{Consistency}
  \begin{block_code}{R code: $N=100$, $1000$, and $10000$}  
\begin{knitrout}\scriptsize
\definecolor{shadecolor}{rgb}{0.969, 0.969, 0.969}\color{fgcolor}\begin{kframe}
\begin{alltt}
\hlcom{#--- Preparation ---#}
\hlstd{B} \hlkwb{<-} \hlnum{1000} \hlcom{# the number of iterations}
\hlstd{N_list} \hlkwb{<-} \hlkwd{c}\hlstd{(}\hlnum{100}\hlstd{,}\hlnum{1000}\hlstd{,}\hlnum{10000}\hlstd{)} \hlcom{# sample size}
\hlstd{N_len} \hlkwb{<-} \hlkwd{length}\hlstd{(N_list)}
\hlstd{estimate_storage} \hlkwb{<-} \hlkwd{matrix}\hlstd{(}\hlnum{0}\hlstd{,B,}\hlnum{3}\hlstd{)} \hlcom{# estimates storage}

\hlkwa{for} \hlstd{(j} \hlkwa{in} \hlnum{1}\hlopt{:}\hlstd{N_len)\{}
        \hlstd{temp_N} \hlkwb{<-} \hlstd{N_list[j]}
        \hlkwa{for} \hlstd{(i} \hlkwa{in} \hlnum{1}\hlopt{:}\hlstd{B)\{}
        \hlcom{#--- generate data ---#}
        \hlstd{x} \hlkwb{<-} \hlkwd{rnorm}\hlstd{(temp_N)} \hlcom{# indep var 1}
        \hlstd{u} \hlkwb{<-} \hlkwd{rnorm}\hlstd{(temp_N)}\hlopt{*}\hlnum{0.2} \hlcom{# error}
        \hlstd{y} \hlkwb{<-} \hlnum{1} \hlopt{+} \hlstd{x} \hlopt{+} \hlstd{u} \hlcom{# dependent variable 1}
        \hlstd{data} \hlkwb{<-} \hlkwd{data.table}\hlstd{(}\hlkwc{y}\hlstd{=y,}\hlkwc{x}\hlstd{=x)}

        \hlcom{#--- OLS ---# }
        \hlstd{reg} \hlkwb{<-} \hlkwd{lm}\hlstd{(y}\hlopt{~}\hlstd{x,}\hlkwc{data}\hlstd{=data)} \hlcom{# OLS}

        \hlcom{#--- store coef estimates ---#}
        \hlstd{estimate_storage[i,j]} \hlkwb{<-} \hlstd{reg}\hlopt{$}\hlstd{coef[}\hlnum{2}\hlstd{]}
        \hlstd{\}}
\hlstd{\}}
\end{alltt}
\end{kframe}
\end{knitrout}
  \end{block_code} 
\end{frame}

\begin{frame}[c,fragile]
	\frametitle{Consistency}
	\begin{block_code}{R code: Visualize}  
\begin{knitrout}\scriptsize
\definecolor{shadecolor}{rgb}{0.969, 0.969, 0.969}\color{fgcolor}\begin{kframe}
\begin{alltt}
\hlcom{#--- wide to long format ---#}
\hlstd{plot_data} \hlkwb{<-} \hlkwd{melt}\hlstd{(}\hlkwd{data.table}\hlstd{(estimate_storage))}

\hlcom{#--- create a figure ---#}
\hlstd{g_co_ex} \hlkwb{<-} \hlkwd{ggplot}\hlstd{(}\hlkwc{data}\hlstd{=plot_data)} \hlopt{+}
        \hlkwd{geom_density}\hlstd{(}\hlkwd{aes}\hlstd{(}\hlkwc{x}\hlstd{=value,}\hlkwc{fill}\hlstd{=variable),}\hlkwc{alpha}\hlstd{=}\hlnum{0.4}\hlstd{)} \hlopt{+}
        \hlkwd{geom_vline}\hlstd{(}\hlkwc{xintercept}\hlstd{=}\hlnum{1}\hlstd{)} \hlopt{+}
        \hlkwd{scale_fill_discrete}\hlstd{(}
                \hlkwc{name}\hlstd{=}\hlstr{'Sample Size'}\hlstd{,}
                \hlkwc{labels} \hlstd{=} \hlkwd{c}\hlstd{(}\hlstr{'N=100 '}\hlstd{,} \hlstr{'N=1,000 '}\hlstd{,}\hlstr{'N=10,000'}\hlstd{)}
                \hlstd{)} \hlopt{+}
        \hlkwd{theme}\hlstd{(}
                \hlkwc{legend.position}\hlstd{=}\hlstr{'bottom'}
                \hlstd{)}
\end{alltt}
\end{kframe}
\end{knitrout}
  \end{block_code} 
\end{frame}

\begin{frame}[c,fragile]
	\frametitle{Consistency}
\begin{knitrout}\footnotesize
\definecolor{shadecolor}{rgb}{0.969, 0.969, 0.969}\color{fgcolor}

{\centering \includegraphics[width=3in,height=3in]{figure/fig_co-1} 

}



\end{knitrout}
\end{frame}

\begin{frame}[c,fragile]
	\frametitle{Consistency}

	\begin{block}{Consistency of OLS estimators}	
		Under $MLR.1$ through $MLR.4$, OLS estimators are consistent
	\end{block}

\end{frame}

\begin{frame}[c,fragile]
  \frametitle{MC simulations: Inconsistency of OLS estimators}
  \begin{block}{Conceptual steps of MC simulations}  
    \begin{itemize}
    \item generate data ($N$ observations) according to $y_i = \beta_0 + \beta_1 x_i + u_i$ with $E[u_i|x_i]\ne 0$
    \item run on the generated data
    \item store the coefficient estimate
    \item repeat the above experiment 1000 times
    \item examine how the coefficient estimates are distributed
  \end{itemize}
  \end{block}
  \only<2->{\begin{block}{What should you see?}  
    Would the bias disappear as N gets larger?
  \end{block}} 
\end{frame}

\begin{frame}[c,fragile]
	\frametitle{Inconsistency}
	\begin{block_code}{R code: $N=100$, $1000$, and $10000$}  
\begin{knitrout}\scriptsize
\definecolor{shadecolor}{rgb}{0.969, 0.969, 0.969}\color{fgcolor}\begin{kframe}
\begin{alltt}
\hlcom{#--- Preparation ---#}
\hlstd{N_list} \hlkwb{<-} \hlkwd{c}\hlstd{(}\hlnum{100}\hlstd{,}\hlnum{1000}\hlstd{,}\hlnum{10000}\hlstd{)} \hlcom{# sample size}
\hlstd{N_len} \hlkwb{<-} \hlkwd{length}\hlstd{(N_list)}
\hlstd{estimate_storage} \hlkwb{<-} \hlkwd{matrix}\hlstd{(}\hlnum{0}\hlstd{,B,}\hlnum{3}\hlstd{)} \hlcom{# estimates storage}

\hlkwa{for} \hlstd{(j} \hlkwa{in} \hlnum{1}\hlopt{:}\hlstd{N_len)\{}
        \hlstd{temp_N} \hlkwb{<-} \hlstd{N_list[j]}
        \hlkwa{for} \hlstd{(i} \hlkwa{in} \hlnum{1}\hlopt{:}\hlstd{B)\{}
        \hlcom{#--- generate data ---#}
        \hlstd{mu} \hlkwb{<-} \hlkwd{rnorm}\hlstd{(temp_N)} \hlcom{# shared term between x and u}
        \hlstd{x} \hlkwb{<-} \hlkwd{rnorm}\hlstd{(temp_N)} \hlopt{+} \hlnum{0.5}\hlopt{*}\hlstd{mu}
        \hlstd{u} \hlkwb{<-} \hlkwd{rnorm}\hlstd{(temp_N)} \hlopt{+} \hlnum{0.5}\hlopt{*}\hlstd{mu}
        \hlstd{y} \hlkwb{<-} \hlnum{1} \hlopt{+} \hlstd{x} \hlopt{+} \hlstd{u} \hlcom{# dependent variable }
        \hlstd{data} \hlkwb{<-} \hlkwd{data.table}\hlstd{(}\hlkwc{y}\hlstd{=y,}\hlkwc{x}\hlstd{=x)}

        \hlcom{#--- OLS ---# }
        \hlstd{reg} \hlkwb{<-} \hlkwd{lm}\hlstd{(y}\hlopt{~}\hlstd{x,}\hlkwc{data}\hlstd{=data)} \hlcom{# OLS}

        \hlcom{#--- store coef estimates ---#}
        \hlstd{estimate_storage[i,j]} \hlkwb{<-} \hlstd{reg}\hlopt{$}\hlstd{coef[}\hlnum{2}\hlstd{]}
        \hlstd{\}}
\hlstd{\}}
\end{alltt}
\end{kframe}
\end{knitrout}
  \end{block_code} 
\end{frame}

\begin{frame}[c,fragile]
  \frametitle{Inconsistency}
  \begin{block_code}{R code: Visualize}  
\begin{knitrout}\scriptsize
\definecolor{shadecolor}{rgb}{0.969, 0.969, 0.969}\color{fgcolor}\begin{kframe}
\begin{alltt}
\hlcom{#--- wide to long format ---#}
\hlstd{plot_data} \hlkwb{<-} \hlkwd{melt}\hlstd{(}\hlkwd{data.table}\hlstd{(estimate_storage))}

\hlcom{#--- create a figure ---#}
\hlstd{g_inco_ex} \hlkwb{<-} \hlkwd{ggplot}\hlstd{(}\hlkwc{data}\hlstd{=plot_data)} \hlopt{+}
  \hlkwd{geom_density}\hlstd{(}\hlkwd{aes}\hlstd{(}\hlkwc{x}\hlstd{=value,}\hlkwc{fill}\hlstd{=variable),}\hlkwc{alpha}\hlstd{=}\hlnum{0.4}\hlstd{)} \hlopt{+}
  \hlkwd{geom_vline}\hlstd{(}\hlkwc{xintercept}\hlstd{=}\hlnum{1}\hlstd{)} \hlopt{+}
  \hlkwd{scale_fill_discrete}\hlstd{(}
    \hlkwc{name}\hlstd{=}\hlstr{'Sample Size'}\hlstd{,}
    \hlkwc{labels} \hlstd{=} \hlkwd{c}\hlstd{(}\hlstr{'N=100 '}\hlstd{,} \hlstr{'N=1,000 '}\hlstd{,}\hlstr{'N=10,000'}\hlstd{)}
    \hlstd{)} \hlopt{+}
  \hlkwd{theme}\hlstd{(}
    \hlkwc{legend.position}\hlstd{=}\hlstr{'bottom'}
    \hlstd{)}
\end{alltt}
\end{kframe}
\end{knitrout}
  \end{block_code} 
\end{frame}

\begin{frame}[c,fragile]
  \frametitle{Inconsistency}
\begin{knitrout}\footnotesize
\definecolor{shadecolor}{rgb}{0.969, 0.969, 0.969}\color{fgcolor}

{\centering \includegraphics[width=3in,height=3in]{figure/fig_inco-1} 

}



\end{knitrout}
\end{frame}

\begin{frame}[c,fragile]
  \frametitle{Inconsistency of OLS estimators}
  \begin{block}{Important}  
    Bias due to the violation of any of the $MLR.1$ through $MLR.4$ would not go away even if you increase the number of observations. 
  \end{block}
\end{frame}

%===================================
% Asymptotic Normality
%===================================
\title{Asymptotic Normality}
\author{}
\date{}

\begin{frame}
\maketitle
\end{frame}

\begin{frame}[c]
  \frametitle{Inference}
  \begin{block}{$MLR.6$: Normality}  
     The population error $u$ is \textcolor{blue}{independent} of the explanatory variables $x_1,\dots,x_k$ and is \textcolor{blue}{normally} distributed with zero mean and variance $\sigma^2$: 
     \begin{align*}
        u\sim Normal(0,\sigma^2) 
     \end{align*}
  \end{block}
  \begin{block}{Remember}  
  \begin{itemize}
  \item If $MLR.6$ are violated, t-statistic and F-statistic we constructed before are no longer distributed as t-distribution and F-distribution, respectively
  \item So, whenever $MLR.6$ is violated, our t- and F-tests are invalid
  \end{itemize}
  \end{block}
\end{frame}

\begin{frame}[c]
  \frametitle{Inference}
  \begin{block}{Fortunately,}  
    You can continue to use t- and F-tests because (slightly transformed) OLS estimators are \textcolor{blue}{approximately} normally distributed when the sample size is \textcolor{blue}{large enough}.
   \end{block} 
\end{frame}

\begin{frame}[c]
  \frametitle{Central Limit Theorem (CLT)}
  \begin{block}{Central Limit Theorem (Lindberg-Levy)}  
    Suppose $\{x_1,x_2,\dots\}$ is a sequence of \textcolor{blue}{identically independently distributed} random variables with $E[x_i]=\mu$ and $Var[x_i]=\sigma^2<\infty$. Then, as $n$ approaches infinity, 
    \begin{align*}
      \sqrt{n}(\frac{1}{n} \sum_{i=1}^n x_i-\mu)\xrightarrow[]{d} N(0,\sigma^2)
    \end{align*}
    Verbally: sample mean less its expected value multiplied by $\sqrt{n}$ is going to be distributed as standard Normal distribution as $n$ goes infinity.
  \end{block}
\end{frame}

\begin{frame}[c]
	\frametitle{CLT}
	\begin{block}{$x_i \sim Bern[p=0.3]$}	
  1 with probability $p$ and 0 with probability $1-p$.
		\begin{itemize}
			\item $E[x_i] = p = 0.3$
			\item $Var[x_i] (\sigma^2) = p(1-p) = 0.21$
		\end{itemize}
	\end{block}

	\begin{block}{According to CLT}	
	\begin{align*}
		\Big(\sqrt{n}(\frac{1}{n} \sum_{i=1}^n x_i-\textcolor{blue}{\mu})\xrightarrow[]{d} N(0,\textcolor{blue}{\sigma^2})\Big) \\
		\sqrt{n}(\frac{1}{n} \sum_{i=1}^n x_i-0.3)\xrightarrow[]{d} N(0,0.21)
	\end{align*}
		
	\end{block}
\end{frame}

\begin{frame}[c,fragile]
  \frametitle{MC simulations: CLT}
  \begin{block}{Conceptual steps of the MC simulation}  
    \begin{itemize}
    \item draw $n$ observations from $x_i\sim Bern(0.3)$ 
    \item find its mean, subtract the expected value (here, $E[x_i]=0.3$), multiply by $\sqrt{n}$ ($\sqrt{n}(\frac{1}{n} \sum_{i=1}^n x_i-\textcolor{blue}{\mu}$)
    \item store the calculated value
    \item repeat the above experiment 1000 times
    \item examine how the calculated values are distributed
  \end{itemize}
  \end{block}
  \only<2->{\begin{block}{What you should see is}  
    As $N$ gets larger (more observations), the distribution of $\sqrt{n}(\frac{1}{n} \sum_{i=1}^n x_i-\textcolor{blue}{\mu}$) looks more and more like $N(0,Var(x_i)$
  \end{block}} 
\end{frame}


\begin{frame}[c,fragile]
  \frametitle{CLT}
  \begin{block_code}{R code: CLT}  
\begin{knitrout}\scriptsize
\definecolor{shadecolor}{rgb}{0.969, 0.969, 0.969}\color{fgcolor}\begin{kframe}
\begin{alltt}
\hlkwd{set.seed}\hlstd{(}\hlnum{893269}\hlstd{)}
\hlcom{#--- the number of observations ---# }
\hlcom{# this is what we change}
\hlstd{N} \hlkwb{<-} \hlnum{10} \hlcom{# number of observations}
\hlstd{B} \hlkwb{<-} \hlnum{1000} \hlcom{# number of iterations}
\hlstd{p} \hlkwb{<-} \hlnum{0.3} \hlcom{# mean of the Bernoulli distribution}
\hlstd{storage} \hlkwb{<-} \hlkwd{rep}\hlstd{(}\hlnum{0}\hlstd{,B)}

\hlkwa{for} \hlstd{(i} \hlkwa{in} \hlnum{1}\hlopt{:}\hlstd{B)\{}
  \hlcom{#--- draw from Bern[0.3] (x distributed as Bern[0.3]) ---#}
  \hlstd{x_seq} \hlkwb{<-} \hlkwd{runif}\hlstd{(N)}\hlopt{<=}\hlstd{p}

  \hlcom{#--- sample mean ---#}
  \hlstd{x_mean} \hlkwb{<-} \hlkwd{mean}\hlstd{(x_seq)}

  \hlcom{#--- normalize ---#}
  \hlstd{lhs} \hlkwb{<-} \hlkwd{sqrt}\hlstd{(N)}\hlopt{*}\hlstd{(x_mean}\hlopt{-}\hlstd{p)}

  \hlcom{#--- save lhs to storage ---#}
  \hlstd{storage[i]} \hlkwb{<-} \hlstd{lhs}
\hlstd{\}}
\end{alltt}
\end{kframe}
\end{knitrout}
  \end{block_code}  
\end{frame}

\begin{frame}[c,fragile]
  \frametitle{CLT Visualization: $N = 10$}
  \begin{block_code}{R code: CLT visualize}
\begin{knitrout}\scriptsize
\definecolor{shadecolor}{rgb}{0.969, 0.969, 0.969}\color{fgcolor}\begin{kframe}
\begin{alltt}
\hlstd{data_pdf} \hlkwb{<-} \hlkwd{data.table}\hlstd{(}
    \hlkwc{x} \hlstd{=} \hlkwd{seq}\hlstd{(}\hlopt{-}\hlnum{2}\hlstd{,}\hlnum{2}\hlstd{,}\hlkwc{length}\hlstd{=}\hlnum{1000}\hlstd{),}
    \hlkwc{y} \hlstd{=} \hlkwd{dnorm}\hlstd{(}\hlkwd{seq}\hlstd{(}\hlopt{-}\hlnum{2}\hlstd{,}\hlnum{2}\hlstd{,}\hlkwc{length}\hlstd{=}\hlnum{1000}\hlstd{),}\hlkwc{sd}\hlstd{=}\hlkwd{sqrt}\hlstd{(p}\hlopt{*}\hlstd{(}\hlnum{1}\hlopt{-}\hlstd{p)))}
    \hlstd{)}
\hlstd{g_N_10} \hlkwb{<-} \hlkwd{ggplot}\hlstd{()} \hlopt{+}
  \hlkwd{geom_density}\hlstd{(}
    \hlkwc{data}\hlstd{=}\hlkwd{data.table}\hlstd{(}\hlkwc{x}\hlstd{=storage),}
    \hlkwd{aes}\hlstd{(}\hlkwc{x}\hlstd{=x,}\hlkwc{color}\hlstd{=}\hlstr{'sample distribution'}\hlstd{)}
    \hlstd{)} \hlopt{+}
  \hlkwd{geom_line}\hlstd{(}
    \hlkwc{data}\hlstd{=data_pdf,}
    \hlkwd{aes}\hlstd{(}\hlkwc{y}\hlstd{=y,}\hlkwc{x}\hlstd{=x,}\hlkwc{color}\hlstd{=}\hlstr{'pdf of N(0,0.21)'}\hlstd{)}
    \hlstd{)} \hlopt{+}
  \hlkwd{scale_color_manual}\hlstd{(}
    \hlkwc{values}\hlstd{=}\hlkwd{c}\hlstd{(}\hlstr{'sample distribution'}\hlstd{=}\hlstr{'blue'}\hlstd{,}\hlstr{'pdf of N(0,0.21)'}\hlstd{=}\hlstr{'red'}\hlstd{),}
    \hlkwc{name}\hlstd{=}\hlstr{''}\hlstd{)} \hlopt{+}
  \hlkwd{theme}\hlstd{(}
    \hlkwc{legend.position}\hlstd{=}\hlstr{'bottom'}
    \hlstd{)} \hlopt{+}
  \hlkwd{ggtitle}\hlstd{(}\hlstr{'N=10'}\hlstd{)}
\end{alltt}
\end{kframe}
\end{knitrout}
  \end{block_code}
\end{frame}

\begin{frame}[c,fragile]
	\frametitle{CLT Visualization: $N = 10$}
\begin{knitrout}\tiny
\definecolor{shadecolor}{rgb}{0.969, 0.969, 0.969}\color{fgcolor}

{\centering \includegraphics[width=3in,height=3in]{figure/clt_viz_10_show-1} 

}



\end{knitrout}
\end{frame}

\begin{frame}[c,fragile]
  \frametitle{CLT Visualization: $N = 100$}
\begin{knitrout}\tiny
\definecolor{shadecolor}{rgb}{0.969, 0.969, 0.969}\color{fgcolor}

{\centering \includegraphics[width=3in,height=3in]{figure/clt_viz_100-1} 

}



\end{knitrout}
\end{frame}

\begin{frame}[c,fragile]
	\frametitle{CLT Visualization: $N = 10,000$}
\begin{knitrout}\tiny
\definecolor{shadecolor}{rgb}{0.969, 0.969, 0.969}\color{fgcolor}

{\centering \includegraphics[width=3in,height=3in]{figure/clt_viz_10000-1} 

}



\end{knitrout}
\end{frame}

\begin{frame}[c]
  \frametitle{CLT}
  \begin{block}{Important}
    CLT holds for \textcolor{blue}{any} distribution of $x_i$ \textcolor{blue}{as long as it has a finite expected value and variance}. 
  \end{block} 
\end{frame}

\begin{frame}[c]
  \frametitle{Asymptotics}

  Under assumptions $MLR.1$ through $MLR.5$ ($MLR.6$ not necessary!!),

  \begin{block}{Asymptotic Normality of OLS}  
    \begin{align*}
    \sqrt{n}(\hat{\beta}_j-\beta_j)\xrightarrow[]{a} N(0,\sigma^2/\alpha_j^2)
    \end{align*}
    where $\alpha_j^2=plim(\frac{1}{n}\sum_{i=1}^n r^2_{i,j})$, where $r^2_{i,j}$ are the residuals from regressing $x_j$ on the other independent variables.
  \end{block}
  
  \begin{block}{Consistency of $\hat{\sigma}^2\equiv \frac{1}{n-k-1}\sum_{i=1}^n \hat{u}_i^2$}  
     $\hat{\sigma}^2$ is a consistent estimator of $\sigma^2$ ($Var(u)$)
   \end{block} 
\end{frame}

\begin{frame}[c]
  \frametitle{Further,}
  For each $j$,
  \begin{itemize}
    \item $(\hat{\beta}_j-\beta_j)/sd(\hat{\beta}_j) \xrightarrow[]{a} N(0,1)$
    \item $(\hat{\beta}_j-\beta_j)/se(\hat{\beta}_j) \xrightarrow[]{a} N(0,1)$, where $se(\hat{\beta}_j)=\sqrt{\frac{\hat{\sigma}^2}{SST_j(1-R_j^2})}$
  \end{itemize}
\end{frame} 

\begin{frame}[c]
  \frametitle{Small vs. Large Sample}
  \begin{block}{Small sample (any sample size)}  
    Under $MLR.1$ through $MLR.5$ \textcolor{blue}{and} $MLR.6$ ($u_i\sim N(0,\sigma^2)$),
    \begin{align*}
      (\hat{\beta}_j-\beta_j)/sd(\hat{\beta}_j) \sim N(0,1)\\
      (\hat{\beta}_j-\beta_j)/se(\hat{\beta}_j) \sim t_{n-k-1}
    \end{align*}
  \end{block} 
  \begin{block}{Large sample (when $n$ goes infinity)}  
    Under $MLR.1$ through $MLR.5$ \textcolor{blue}{without} $MLR.6$,
    \begin{align*}
     (\hat{\beta}_j-\beta_j)/sd(\hat{\beta}_j) \xrightarrow[]{a} N(0,1) \\
     (\hat{\beta}_j-\beta_j)/se(\hat{\beta}_j) \xrightarrow[]{a} N(0,1) 
    \end{align*}
  \end{block}
\end{frame}

\begin{frame}[c]
  \frametitle{Testing under large sample}
  \begin{block}{It turns out,}  
    You can proceed exactly the same way as you did before (practically speaking)!!

    \begin{enumerate}
      \item calculate $(\hat{\beta}_j-\beta_j)/se(\hat{\beta}_j)$
      \item check if the obtained value is greater than (in magnitude) the critical value for the specified significance level under $t_{n-k-1}$
    \end{enumerate}
  \end{block}
  \only<2->{\begin{block}{But,}  
    Shouldn't we use $N(0,1)$ when you find the critical value?
  \end{block}}
\end{frame}

\begin{frame}[c,fragile]
  \frametitle{Testing under large sample}
  \begin{block_code}{R code: t vs N distributions}
\begin{knitrout}\tiny
\definecolor{shadecolor}{rgb}{0.969, 0.969, 0.969}\color{fgcolor}\begin{kframe}
\begin{alltt}
\hlstd{x} \hlkwb{<-} \hlkwd{seq}\hlstd{(}\hlopt{-}\hlnum{3}\hlstd{,}\hlnum{3}\hlstd{,}\hlkwc{length}\hlstd{=}\hlnum{1000}\hlstd{)}
\hlstd{y_norm} \hlkwb{<-} \hlkwd{dnorm}\hlstd{(x)} \hlcom{# pdf of N(0,1)}
\hlstd{y_t_2} \hlkwb{<-} \hlkwd{dt}\hlstd{(x,}\hlkwc{df}\hlstd{=}\hlnum{2}\hlstd{)} \hlcom{# pdf of t_\{2\}}
\hlstd{y_t_10} \hlkwb{<-} \hlkwd{dt}\hlstd{(x,}\hlkwc{df}\hlstd{=}\hlnum{10}\hlstd{)} \hlcom{# pdf of t_\{10\}}
\hlstd{y_t_30} \hlkwb{<-} \hlkwd{dt}\hlstd{(x,}\hlkwc{df}\hlstd{=}\hlnum{30}\hlstd{)} \hlcom{# pdf of t_\{30\}}
\hlstd{y_t_50} \hlkwb{<-} \hlkwd{dt}\hlstd{(x,}\hlkwc{df}\hlstd{=}\hlnum{50}\hlstd{)} \hlcom{# pdf of t_\{50\}}

\hlstd{plot_data} \hlkwb{<-} \hlkwd{data.table}\hlstd{(}
  \hlkwc{x}\hlstd{=x,}
  \hlstr{'N(0,1)'}\hlstd{=y_norm,}
  \hlstr{'t (df=2)'}\hlstd{=y_t_2,}
  \hlstr{'t (df=10)'}\hlstd{=y_t_10,}
  \hlstr{'t (df=30)'}\hlstd{=y_t_30,}
  \hlstr{'t (df=50)'}\hlstd{=y_t_50}
  \hlstd{)} \hlopt
  \hlkwd{melt}\hlstd{(}\hlkwc{id.var}\hlstd{=}\hlstr{'x'}\hlstd{)}

\hlstd{g_t_vs_N} \hlkwb{<-} \hlkwd{ggplot}\hlstd{(}\hlkwc{data}\hlstd{=plot_data)} \hlopt{+}
  \hlkwd{geom_line}\hlstd{(}\hlkwd{aes}\hlstd{(}\hlkwc{y}\hlstd{=value,}\hlkwc{x}\hlstd{=x,}\hlkwc{color}\hlstd{=variable))} \hlopt{+}
  \hlkwd{scale_color_discrete}\hlstd{(}\hlkwc{name}\hlstd{=}\hlstr{''}\hlstd{)} \hlopt{+}
  \hlkwd{theme}\hlstd{(}
    \hlkwc{legend.position}\hlstd{=}\hlstr{'bottom'}
    \hlstd{)}
\end{alltt}
\end{kframe}
\end{knitrout}
  \end{block_code}
\end{frame}

\begin{frame}[c]
  \frametitle{t vs Normal distributions}
\begin{knitrout}\tiny
\definecolor{shadecolor}{rgb}{0.969, 0.969, 0.969}\color{fgcolor}

{\centering \includegraphics[width=3in,height=3in]{figure/t_n_viz-1} 

}



\end{knitrout}
\end{frame}

\begin{frame}[c]
  \frametitle{Testing under large sample}
  Since $t_{n-k-1}$ and $N(0,1)$ are almost identical when $n$ is large, there is very little error in using $t_{n-k-1}$ instead of $N(0,1)$ to find the critical value.
\end{frame}

\begin{frame}[c]
  \frametitle{Homoskedasticity}
  \begin{block}{Important}  
  The asymptotic normality of OLS \textcolor{blue}{does} require homoskedasticity assumption ($MLR.5$)!!

  \begin{itemize}
    \item the usual t-statistics and confidence intervals are invalid no matter how large the sample size is if error is heteroskedastic
    \item we talk extensively about how we should deal with heteroskedasticity
  \end{itemize}
  \end{block} 
\end{frame}

\end{document}





